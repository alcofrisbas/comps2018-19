\section{Description}\label{sec:description}
To generate WriterBot’s suggestions, we will use a recurrent neural network based
language model (RNN LM). RNNs are very efficient in predicting results in sequential
data. In particular, we will be using Long Short Term Memory network which is a
special kind of RNN, capable of learning long term dependencies. They allow
information to cycle inside the network for a very long time, allowing access
to more contextual information than just the current sentence to predict the next
sentence.

WriterBot will be trained on a library of popular public-domain novels. It will
be
constructed as a RNN LM that will take input from both the sentence that the dataset
has input (also known as the context) and the words that it has generated.
\cite{RNN_language_model}. Once the sentence has been processed and the output
generated, the weights attributed to each RNN cell will be updated according to
the loss and how accurately the cell predicted the correct output.

When the user interacts with the trained RNN their sentence will function like
the dataset does, as the context, for each RNN cell and each cell will feed into
the next with the word it has generated. When this process is completed, the RNN
will have output a complete sentence that the user can then edit or delete at their
leisure. The user will then type another sentence and the RNN will take the new
sentence as input and follow the same process.

We will use a simple recurrent neural network architecture (Mikolov et al., 2010),
which is relatively easy to implement and train. We will also explore different
optimization techniques to improve the performance of the neural network like
optimizing the learning rate schedule. 
